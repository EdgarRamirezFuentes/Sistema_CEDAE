\documentclass[12pt,letterpaper]{article}

\usepackage[utf8]{inputenc}
\usepackage{ragged2e}
\usepackage{amsfonts}
\usepackage{amssymb}
\usepackage{graphicx}
\usepackage{multicol}
\usepackage{changepage}
\usepackage{float}
\usepackage{cite}
\usepackage{url}
\usepackage[left=2.50cm, right=2.50cm]{geometry}
\usepackage[spanish]{babel}


\author{Ram\'irez Fuentes Edgar Alejandro}
\title{Entrega 1}
\date {2020-10-03}

\begin{document}
	%encabezado 
	\pagestyle{plain}
	{

		% Imágenes de la portada
		{
			\begin{tabular}
				{
					p{0.75\textwidth} 
					p{0.25\textwidth} 
				}
				\includegraphics[width=1.5cm, height=2.5cm]{ipn.png} &  
				\includegraphics[width=2.5cm, height=2cm]{escom.png}
			\end{tabular}
		}

		% Datos de la caratula
		\begin{center}

			\par\vspace{1cm} %Espacio dejado antes del encabezado
			{
				\Huge\textbf
				{
					Instituto Polit\'ecnico Nacional 
					\\[.2cm]Escuela Superior de C\'omputo
				}
			}

			\par\vspace{0.5cm}
			{
				\Large\textbf
				{
					Ingenier\'ia en sistemas computacionales 
					\\[.5cm]An\'alisis y diseño orientado a objetos
				}
			}

			\vfill

			\par\vspace{0.5cm}
			{
				\Large\textbf
				{
					Entrega 1 \\
					CEDAE (Centro de dematológico de alta especialidad)
				}
			}

			\vfill

			\par\vspace{1cm}
			{
				\large\textbf
				{
                    Equipo 3:
                    \\Angeles Hernández Jesús Eduardo
                    \\Chanes Nuñez Ricardo Jehonadab
                    \\García Gamiño Rafael Julian
                    \\Hernández Ceciliano Luis Ángel
                    \\Mendoza Cuellar José Oscar
                    \\Olvera Olvera Kevin Jesús
                    \\Paniagua Juárez Nadia Patricia
                    \\Ramírez Fuentes Edgar Alejandro
                    \\Zamorano Cruz Juan Raymundo
					\\2CV9
				} 
			}

			\par\vspace{3cm}

		\end{center}
		\clearpage
	}

	\newpage
	\pagestyle{plain}
	{
		\begin{center}
			\par\vspace{0.5cm}
			{
				\Huge\textbf
				{
					\'Indice
				}
			}
		\end{center}
		\par\vspace{0.5cm}
			{
				\Large\textbf
				{
					\begin{itemize}
						\item Organización del SCRUM Team ... 3
                        \begin{itemize}
                            \item Product Owner
                            \item SCRUM Master
                            \item Development team
                        \end{itemize}
						\item Acuerdos del SCRUM Team ... 4
						\item Descripción del proyecto ... 5
						\item Requerimientos del proyecto ... 6
                            \begin{itemize}
                                \item Funcionales
                                \item No funcionales
                            \end{itemize}
					\end{itemize}
				}
			}

			\vfill
	}

	\newpage
	\pagestyle{plain}
	{
		\par\vspace{0cm}
		{
			\begin{center}
					\Huge\textbf
					{
						Organización del SCRUM Team
					}
			\end{center}
		}
		\par\vspace{0cm}
		{
			\normalsize\textbf
			{
				\justify
                Después de una reunión enfocada a la selección de los roles
                que tendrá cada integrante del equipo, se llegó a la siguiente
                organización.
                \\\\
                \begin{itemize}
                    \item Product Owner - M.C. Maldonado Castillo Idalia 
                    \item SCRUM Master - Hernández Ceciliano Luis Ángel 
                    \item Development team
                        \begin{itemize}
                            \item Equipo de análisis
                                \begin{itemize}
                                    \item Angeles Hernández Jesús Eduardo
                                    \item Hernández Ceciliano Luis Ángel
                                \end{itemize}
                            \item Equipo de diseño
                            \begin{itemize}
                                \item Panigua Juárez Nadia Patricia
                                \item Zamorano Cruz Juan Raymundo
                            \end{itemize}
                            \item Equipo de desarrollo
                            \begin{itemize}
                                \item Bases de Datos
                                    \begin{itemize}
                                        \item Mendoza Cuellar José Oscar
                                        \item Olvera Olvera Kevin Jesús
                                    \end{itemize}
                                \item Front-end
                                \begin{itemize}
                                    \item Ramírez Fuentes Edgar Alejandro
                                    \item Panigua Juárez Nadia Patricia
                                    \item Zamorano Cruz Juan Raymundo
                                \end{itemize}
                                \item Back-end
                                \begin{itemize}
                                    \item Chanes Nuñez Ricardo Jehonadab
                                    \item García Garmiño Rafael Julian
                                    \item Ramírez Fuentes Edgar Alejandro
                                \end{itemize}
                            \end{itemize}
                        \end{itemize}
                \end{itemize}
			}
		}
    }
    
    \newpage
	\pagestyle{plain}
	{
		\par\vspace{0cm}
		{
			\begin{center}

					\Huge\textbf
					{
						Acuerdos del SCRUM Team
					}
			\end{center}
		}
		\par\vspace{0cm}
		{
			\normalsize\textbf
			{
                \justify
                En una de las reuniones diarias del SCRUM se llegaron a
                los siguientes acuerdos, los cuales deberán ser cumplidos de 
                manera exacta.
                \begin{itemize}
                    \item Daily standup meetings
                    \begin{itemize}
                        \item Las daily meeting se realizarán de Lunes a Viernes en un horario de 22:00 a 22:18 
                        \item Cada participante tendrá 2 minutos para hablar
                        \item Cada integrante deberá tener listos los puntos a tratar antes de la reunión para evitar prolongar la reunión
                        \item No se tendrá tolerancia ante retardos
                        \item Si un integrante tiene más de 5 retardos o faltas en estas reuniones, se expulsará del equipo por falta de compromiso con el proyecto
                        \item Solo se tocarán temas con respecto a actividades realizadas el día de la reunión y obstáculos que se tienen para la realización de futuras actividades
                        \item Si los equipos desean reunirse por su parte para tratar dudas acerca del proyecto, son libres de hacerlo.
                    \end{itemize}
                    \item Sprint planning meetings, Sprint review meeting y Sprint retrospective meeting
                    \begin{itemize}
                        \item Este tipo de reuniones se realizarán en los días sabádos para la dedicación de tiempo necesario.
                        \item No habrá tolerancia ante retrasos en este tipo de reuniones
                        \item No habrá tolerancia ante ausencia en este tipo de reunión. Si algún integrante llega a faltar, se pondrá a votación su estancia en el equipo de trabajo 
                        \item La duración de la reunión será dependiente del entendimiento organización, opiniones del equipo acerca del proyecto 
                    \end{itemize}
                \end{itemize}
			}
		}
    }
    
    \newpage
	\pagestyle{plain}
	{
		\par\vspace{0cm}
		{
			\begin{center}

					\Huge\textbf
					{
						Descripción del proyecto
					}
			\end{center}
		}
		\par\vspace{0cm}
		{
			\normalsize\textbf
			{
                \justify
                Nota: La descripción actual del proyecto es con base en la información que hasta el 
                momento se tiene acerca del proyecto.
                \\\\\\
                Se desarrollará un sistema web el cual apoyará a la clínica CEDAE a la gestión de doctores titulares,
                doctores auxiliares, trabajadores de la clínica, pacientes, stock de medicina, citas médicas, expedientes médicos,
                recetas médicas, entre otras futuras funcionalidades.
                Además, se implementarán todos aquellos servicios y subservicios que la clínica CEDAE presta para así lograr satisfacer
                las necesidades del cliente.
                \\\\
                El sistema le permitirá al médico titular visualizar expedientes de sus pacientes, emitir recetas tanto físicas,
                por correo electrónico o por la plataforma web, permitir el acceso a su receta médica de sus pacientes, entre otras futuras funcionalidades.
                \\\\
                El sistema permitirá a los usuarios, únicamente aquellos registrados en la plataforma, visualizar recetas previamente autorizadas por 
                su médico, visualizar futuras citas, entre otras futuras funcionalidades.
                \\\\
                Para el módulo del área de farmacia se llevará el control de los medicamentos en stock, se dará alerta de aquellos medicamentos que estén cerca de incumplimiento 
                de normas de COFEPRIS para así ayudar al cliente a priorizar la venta de medicamento cercano a ser desechado, entre otras futuras funcionalidades.
                \\\\
                Por parte del sector administrativo de la clínica permitirá generar citas, generar reportes estadísticos que el cliente requiera, entre futuras funcionalidades.
                \\\\
                Cabe recalcar que esta no es una descripción final del sistema, esta está sujeta a cambios con base a las necesidades del cliente. 
			}
		}
    }
    
    \newpage
	\pagestyle{plain}
	{
		\par\vspace{0cm}
		{
			\begin{center}

					\Huge\textbf
					{
						Requerimientos del proyecto
					}
			\end{center}
		}
		\par\vspace{0cm}
		{
			\normalsize\textbf
			{
                \justify
                Nota: Los siguientes requirimientos están sujetos a cambios, por lo cual no se deberán tomar como requerimientos finales.
                \\\\\\\\
                \begin{itemize}
                    \item Requerimientos funcionales
                    \begin{enumerate}
                        \item Cada usuario podrá iniciar y cerrar sesión en el sistema.
                        \item El usuario administrador podrá registrar otros usuarios.
                        \item El usuario recepcionista podrá agendar citas para pacientes.
                        \item El paciente podrá consultar las fechas de sus citas a través de su cuenta.
                        \item El sistema permitirá al médico titular generar recetas médicas.
                        \item El médico titular podrá generar o hacer cambios al expediente clínico de pacientes.
                        \item Los médicos titulares y auxiliares podrán consultar expedientes clínicos de sus pacientes.
                        \item El paciente podrá consultar sus expediente clínico a través de su cuenta.
                        \item El médico titular podrá habilitar los expedientes clínicos para su consulta por parte del usuario.
                        \item El usuario administrador podrá generar de reportes estadísticos.
                        \item El sistema contará con manejo de almacén de farmacia.
                        \item Los médicos podrán consultar la disponibilidad de medicamentos en farmacia.
                        \item El sistema generará una advertencia para los usuarios encargados de farmacia cuando un medicamento esté pronto a caducar.
                        \item Los médicos podrían generar recetas físicas o electrónicas.
                        \item Cada consulta se llevará un registro de el personal que atendió al paciente.
                    \end{enumerate}
                    \item Requerimientos no funcionales
                    \begin{enumerate}
                        \item Los expedientes clínicos serán generados con base en la norma NOM-024-SSA3-2010.
                        \item El paciente podrá ver su expediente médico únicamente si ha pedido autorización.
                        \item Las caducidades de medicamentos serán tratadas conforme a las normas de la COFEPRIS.
                        \item El surtido de recetas por parte de la farmacia se debe apegar al Artículo 226 de la Ley General de Salud.
                    \end{enumerate}
            \end{itemize}
			}
		}
	}
\end{document}
